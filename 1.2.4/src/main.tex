%! Author = admin
%! Date = 07.12.2022

% Preamble
\documentclass[a4paper,12pt]{article}

% Packages
\usepackage{multirow}
\usepackage{wrapfig}
\usepackage[T2A]{fontenc}
\usepackage[utf8]{inputenc}
\usepackage[english,russian]{babel}
\usepackage{indentfirst}
\usepackage[usenames]{color}
\usepackage[a4paper,top=1cm,bottom=1cm,left=1cm,right=1cm,marginparwidth=0.75cm]{geometry}
\usepackage{colortbl}
\usepackage{float}
\usepackage{graphicx}
\usepackage{wrapfig}
\usepackage{hyperref}
\usepackage[rgb]{xcolor}
\usepackage{todonotes}
\usepackage{amsmath,amsfonts,amssymb,amsthm,mathtools}
\usepackage{makecell}


% package to open file containing variables
\usepackage{datatool, filecontents}
\usepackage{colortbl}
\usepackage{booktabs}
\DTLsetseparator{,}% Set the separator between the columns.

% import data
\DTLloadrawdb[noheader, keys={thekey,thevalue}]{output_data}{C:/labas/1.2.4/data/output_data.csv}
% Loads output_data.csv with column headers 'thekey' and 'thevalue'
\newcommand{\var}[1]{\DTLfetch{output_data}{thekey}{#1}{thevalue}}

% Document
\begin{document}

    \begin{centering}
        \section* {{\fboxsep=0pt\colorbox{green!50}{\strut Обработка данных:}}}
    \end{centering}

    \begin{itemize}

        \item {Размеры тел}

        \begin{table}[h!]
            \centering
            \begin{tabular}{|c|c|c|c|c|c|c|c|c|}
                \hline
                & $a_{\text{к}}$ & $a$ & $b$ & $c$ & $h_{\text{ц}}$ & $r_{\text{ц}}$ & $h_{\text{д}}$ & $r_{\text{д}}$
                \\ \hline
                значение, см & \var{ak} & \var{a} & \var{b} & \var{c} & \var{hц} & \var{rц} & \var{hд} & \var{rд}
                \\ \hline
                $\varepsilon$ &  \var{reak} & \var{rea} & \var{reb} & \var{rec} & \var{rehц} & \var{rerц} & \var{rehд} & \var{rerд}
                \\ \hline
            \end{tabular}
            \caption{Размеры исследуемых тел и их погрешности}
        \end{table}

        Абсолютная погрешность измерения размеров линейкой составляет 0.005 см.

        \item{Периоды колебаний}

        \begin{table}[h!]
            \centering
            \begin{tabular}{|c|c|c|c|c|c|c|c|c|c|c|}
                \hline
                $T$ & $T_{\text{1z}}$ & $T_{\text{1x}}$ & $T_{\text{1y}}$ & $T_{\text{1d}}$ & $T_{\text{1e}}$ & $T_{\text{1p}}$ & $T_{\text{1m}}$ & $T_{\text{2x}}$ & $T_{\text{2y}}$ & $T_{\text{2z}}$
                \\ \hline
                $T_{\text{cp}}$, c & \var{T1z} & \var{T1x} & \var{T1y} & \var{T1d} & \var{T1e} & \var{T1p} & \var{T1m} & \var{T2x} & \var{T2y} & \var{T2z}
                \\ \hline
                $\sigma_{\text{случ}}$, c & \var{rdeT1z} & \var{rdeT1x} & \var{rdeT1y} & \var{rdeT1d} & \var{rdeT1e} & \var{rdeT1p} & \var{rdeT1m} & \var{rdeT2x} & \var{rdeT2y} & \var{rdeT2z}
                \\ \hline
                $\sigma_{\text{полн}}$, c & \var{feT1z} & \var{feT1x} & \var{feT1y} & \var{feT1d} & \var{feT1e} & \var{feT1p} & \var{feT1m} & \var{feT2x} & \var{feT2y} & \var{feT2z}
                \\ \hline
                $\varepsilon_{\text{полн}}$ & \var{reT1z} & \var{reT1x} & \var{reT1y} & \var{reT1d} & \var{reT1e} & \var{reT1p} & \var{reT1m} & \var{reT2x} & \var{reT2y} & \var{reT2z}
                \\ \hline
            \end{tabular}


            \begin{tabular}{|c|c|c|c|c|c|c|c|c|c|}
                \hline
                $T$ & $T_{\text{2d}}$ & $T_{\text{2e}}$ & $T_{\text{2m}}$ & $T_{\text{2p}}$ & $T_{\text{p}}$ & $T_{\text{3x}}$ & $T_{\text{3y}}$ & $T_{\text{4x}}$ & $T_{\text{4y}}$
                \\ \hline
                $T_{\text{cp}}$, c & \var{T2d} & \var{T2e} & \var{T2m} & \var{T2p} & \var{Tp} & \var{T3x} & \var{T3y} & \var{T4x} & \var{T4y}
                \\ \hline
                $\sigma_{\text{случ}}$, c & \var{rdeT2d} & \var{rdeT2e} & \var{rdeT2m} & \var{rdeT2p} & \var{rdeTp} & \var{rdeT3x} & \var{rdeT3y} & \var{rdeT4x} & \var{rdeT4y}
                \\ \hline
                $\sigma_{\text{полн}}$, c & \var{feT2d} & \var{feT2e} & \var{feT2m} & \var{feT2p} & \var{feTp} & \var{feT3x} & \var{feT3y} & \var{feT4x} & \var{feT4y}
                \\ \hline
                $\varepsilon_{\text{полн}}$ & \var{reT2d} & \var{reT2e} & \var{reT2m} & \var{reT2p} & \var{reTp} & \var{reT3x} & \var{reT3y} & \var{reT4x} & \var{reT4y}
                \\ \hline
            \end{tabular}
            \caption{Средние значения периодов колебаний ($T_{\text{cp}}$) и их погрешности}
        \end{table}

        Систематическая погрешность для всех измерений одинакова и складывается из погрешности секундомера и скорости
        реакции экспериментатора, которая определяется с помощью измерения временного промежутка между двумя нажатиями
        на кнопку. В моем случае скорость реакции составляет 0.13 с, а погрешность секундомера - 0.001 с, следовательно
        ей можно пренебречь и принять систематическую погрешность равной 0.13 с.


    \end{itemize}
\end{document}
